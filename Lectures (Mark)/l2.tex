\section{Image Sampling}

\subsection{Sampling Signals}
Given a continuos signal, an image is well sampled if from the sample points alone, an accurate representation of the original signal can be created. Bad sampling would describe a different signal upon creation.
\\
If we were to oversample, lets use the example of a rotating wheel, the wheel would appear to be rotating slowly. If the sampling rate was not high enough, the wheel would be appear to be moving in the opposite direction.
\\
Consider the frequency domain, the spectra repeat, if the sampling rate is correct, the spectra will just touch. We can deduce from this that:
\begin{equation}
    Sample_{min} = 2 \times f_{max}
\end{equation}
This follows \textit{Nyquists sampling theorem}, which states:
\begin{quote}
    In order to be able to reconstruct a singal from its samples, we must sample at minimum twice the maximum frequency in the original sample.
\end{quote}
A simple example of this would be, speech averages at $6kHz$, we sample at $12kHz$. If we compare this to something like \textit{.wav} audio files, they are often sampled at 24kHz upon a spectral analysis. In image terms; two pixels for every pixel of interest.

\subsection{1D Discrete Fourier Transform}
Discrete Fourier calculates frequency from data points.

\begin{align}
    \mathcal{F}p(\omega) = \int_{-\infty}^{\infty}p(t)e^{-jwt}dt \\
    \mathcal{F}p_{u} =\frac{1}{N}\sum_{i=0}^{N-1}p_{i}e^{-j\frac{2\pi}{N}iu} \\
\end{align}

Where:
\begin{align}
    \text{Sampled Frequency: } \mathcal{F}p_{u} \\
    \text{Sampled Points: } p_{i}\\
    N \text{ points}
\end{align}

Again, for a 10 point sample ($N=10$), we can reconstruct these signals by adding all frequency components:

\begin{align}
    \mathcal{F}p_{6} = \sum_{u=0}^{9}\mathcal{F}p_{u}\times e^{jt\frac{2\pi}{10}u}
\end{align}
\newpage

\subsection{2D Discrete Fourier Transform}
The forward and inverse transforms are given:

\begin{align}
    \mathbf{\mathcal{F} P_{u,v}} = \frac{1}{N^{2}}\sum_{x=0}^{N-1}\sum_{y=0}^{N-1}\mathbf{P_{x,y}}e^{-j(\frac{2\pi}{N})(ux+vy)} \\
    \mathbf{\mathcal{F}^{-1}} = \sum_{u=0}^{N-1}\sum_{v=0}^{N-1}\mathbf{\mathcal{F}P_{u,v}}e^{j(\frac{2\pi}{N}(ux+vy)}\\
    \text{two dimensions of space, x and y} \\
    \text{two dimensions of frequency, u and v} \\
    \text{image $NxN$, pixels each being } \mathbf{P_{x,y}} 
\end{align}

\begin{quote}
    \textbf{Shift Invariance} Upon applying a (positional) shift to an image, the magnitude and phase of the image are invariant to the shifted image. This is only true if all pixels overflow as a result of a positional shift.
\end{quote}

We can apply a few different transforms after applying the 2D Fourier transform to an image. For example rotation and filtering, mathematically we can rotate an image 90 degrees with the following transform.

\begin{equation}
    \mathbf{\mathcal{F} P_{u,v}} = \frac{1}{N}\sum_{x=0}^{N-1}\sum_{y=0}^{N-1}\mathbf{P_{x,y}}e^{-j(\frac{2\pi}{N})(ux+vy)} \\
\end{equation}

Given we have access to all frequencies for an image, we can filter out certain frequencies to create a low pass filtered image. This can then be used for different applications such as creating hybrid images.
\\
Some other transforms include: Fourier transform magnitude, discrete cosine transform, hartley transform.

\subsubsection{Applications of 2D Fourier Transform}
\begin{itemize}
    \itemsep0em
    \item Understanding and analysis
    \item Speeding up algorithms
    \item Invariance
    \item Coding
    \item Recognition of frequency (e.g. texture)
\end{itemize}

