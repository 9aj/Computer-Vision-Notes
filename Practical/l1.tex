\section{Practical Section}

\subsection{Task}
Classify social media posts with images or videos as real or fake.

Definition of a fake post:
- reposting real multimedia as being associated to a current event
- fake images
- synthetic images

1 algorithm choice with justification

F1 scores reported for each algorithm test (parameters)
Software includes install and run guide

Metric - F1 score (but you might want to report P, R, ROC ...)
‐ Binary class labels == 'fake' (positive) or 'real' (negative)
‐ TP >> Classified fake + Ground truth fake
‐ FP >> Classified fake + Ground truth real
‐ TN >> Classified real (or unknown) + Ground truth real
‐ FN >> Classified real (or unknown) + Ground truth fake

 Consider enriching the data
‐ No image features
‐ You can use external generic data
NLTK stopwords, lists of common first names, lists of respected news
organizations, sentiment word lists

‐ You can pre-process your data and make new features
TF-IDF, Stanford POS and NER taggers, VADER sentiment analysis
‐ What is the humour label?

Humour label should be treated as a Fake label for eval. It might be
helpful for training (or it might not!).
‐ You cannot edit the test dataset to make it easier
F1 scores must be run on full test dataset to allow a fair comparison of results

Show us things like graphs, histograms, confusion matrix,
ranked lists of top features, skew in data segments, gap analysis ...
• Marking scheme (COMP 3222 - UG) ‐ Introduction and data analysis (20)
The lab on "Data analysis and visualization" will help here