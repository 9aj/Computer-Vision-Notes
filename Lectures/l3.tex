\section{Point Operators}

\subsection{Histogram Representation}
We can display the contrast of an image by simply plotting a histogram of the pixels with a certain brightness value, and the frequency.
\\
As a result, we can change the contrast by applying a brightness function:

\begin{align}
    N_{x,y} = k\times \mathbf{O}_{x,y}+l \\
    N_{x,y}\text{ New image, pixels x and y} \\
    \mathcal{O}_{x,y}\text{ old image} \\
    \text{gain k} \\
    \text{level l}
\end{align}

\subsection{Logarithmic Compression...}
We can apply logarithmic compression and exponential expansion.
\begin{align}
    N_{x,y} = log(\mathcal{O}_{x,y}) \\
    N_{x,y} = e^{(\mathcal{O}_{x,y})}
\end{align}

\subsection{Intensity Normalisation}

The aim is to use all available grey levels for display. We can do this using the following process:
\begin{itemize}
    \itemsep0em
    \item [1] Take the original histogram
    \item [2] Shift the origin to zero
    \item [3] Scale brightness to use the whole range
\end{itemize}

This can be described using the following equation:

\begin{align}
    N_{x,y} = \frac{N_{max} - N_{min}}{O_{max}-O_{min}} \times (O_{x,y} - O_{min}) + N_{min} \\
    N_{x,y} = \frac{256}{O_{max}-O_{min}} \times (O_{x,y} - O_{min})
\end{align}

The second equation is used to avoid the need of parameter selection.
\newpage
\subsection{Histogram Equalisation}
We can flatten a histogram, which is aimed for human vision to show more detail (e.g. for medical applications).

\begin{table}[ht!]
\centering
{%
\begin{tabular}{lr}
 $N^{2}$ points, the sum of points per level is equal & $\sum_{l=0}^{M}\mathbf{O}(l) = \sum_{l=0}^{M}N(l)$ \\[0.5cm]
 cumulative histogram up to level p transformed & $\sum_{l=0}^{p}\mathbf{O}(l) = \sum_{l=0}^{q}N(l)$  \\[0.5cm]
 Number of points per level in output picture & $N(l)=\frac{N^{2}}{N_{max}-N_{min}}$ \\[0.5cm]
 cumulative histogram of output & $\sum_{l=0}^{q}N(l) = q \times N(l)=\frac{N^{2}}{N_{max}-N_{min}}$  \\[0.5cm]
 mapping for the output pixels at level q & $q = \frac{N_{max} - N_{min}}{N^{2}}\times \sum_{l=0}^{p} \mathbf{O}(l)$ 
\end{tabular}%
}
\end{table}

This is nonlinear and there are major problems with noisy images.

\subsection{Thresholding}
Thresholding selects points that exceed a chosen threshold

\begin{equation}
    N_{x,y} = \begin{cases}
                255 & \text{if } N_{x,y}>threshold \\
                0 & \text{otherwise}
                \end{cases}
\end{equation}

\subsection{Summary}
\begin{itemize}
    \itemsep0em
    \item [1] point operators are largely about image display
    \item [2] histogram manipulation
    \item [3] thresholding is used a lot
    \item [4] intensity normalisation is used for medical display
\end{itemize}