\section{Image Classification and Auto-annotation}

This lecture is largely aimed as supplementary material for the final coursework. Only examinable contents are provided.

\subsection{Multilabel Classification}

Where we have a cat and a dog, as labels, in the context of images this is often called \textbf{Automatic Annotation}.

\subsection{Dense SIFT}
Rather than extracting SIFT features at difference of Gaussian interest points, we could extract them across a dense grid. This gives more coverage of the entire image.

\subsection{Pyramid Dense SIFT}
For even better performance, we can use the scaled Gaussian pyramid. The sampling region is a fixed size, so at higher scales you sample more content.

\subsection{Developing and Benchmarking a BoVW scene classifier}

Again, this appears to be relevant only for the coursework but is included as a guide.

\begin{itemize}
    \itemsep0em
    \item [\textbf{1}] \textbf{Evaluation Dataset}
    \item []We split the dataset into testing and training data
    \item []Only training data is used to train the annotator
    \item [\textbf{2}] \textbf{Building the BoVW}
    \item []Extract raw image features from training images
    \item []Learn a codebook from these features
    \item []Use K Means from a sample
    \item []Apply vector quantisation to assign histograms from each image.
    \item [\textbf{3}] \textbf{Training Classifiers}
    \item []Classifier can be trained using the histograms
    \item []Train on a subset of training data and create validation set with remaining data. This is for the purpose of parameter optimisation.
    \item [\textbf{4}] \textbf{Classifying the Test Set}
    \item []Use classifiers to find the most likely class
    \item [\textbf{5}] \textbf{Calculate different metrics for performance evaluation}
    \item []Average precision
    \item []Recall
    \item []F1 macro/micro score
\end{itemize}

\hline
\vfill

\begin{quote}
    \centering
    The End.
    7600 Words.
    
    Produced for COMP3204 Computer Vision
    
    \textit{Alex Miles / 9aj}
\end{quote}
